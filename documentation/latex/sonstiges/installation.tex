\section{Installation}
Windows-Nutzer k�nnen weitgehende Installationsschritte durch die Verwendung unserer \emph{Windows One-Click Distribution} in der beigelegten Datei \texttt{shoppingtour-w32.zip} vermeiden. Hierzu wird diese Datei in einen beliebigen Ordner entpackt und die enthaltene \texttt{shoppingtour.exe}-Datei ausgef�hrt.

Da dieses Projekt ausschlie�lich in der plattformunabh�ngigen Skriptsprache \emph{Python} geschrieben wurde, ist eine Distribution auf fast alle Betriebssysteme m�glich. Notwendige Voraussetzung hierf�r ist jedoch, dass die im Projekt verwendeten notwendigen Bibliotheken f�r das Zielsystem verf�gbar sind. Da dies meist nur f�r verbreitete Betriebssysteme der Fall ist, k�nnen wir eine Lauff�higkeit zum aktuellen Zeitpunkt nur f�r \emph{Microsoft Windows} (2000, XP, Vista, 7, 8 Developer Preview), \emph{Apple Mac OS X} (>= 10.4) und \emph{Linux} (Kernel 2.6/3.0, insbesondere Distributionen ''Debian'', ''Ubuntu'' und ''Gentoo'') garantieren.

Im Folgenden werden die notwendigen Softwarepakete aufgef�hrt, die f�r die Verwendung von \emph{ShoppingTour} erforderlich sind.

\subsection{ShoppingTour}
\url{http://www.myhpi.de/~kai.fabian/shoppingtour/}

ShoppingTour ben�tigt keine Installation und kann nach dem Entpacken in ein beliebiges Verzeichnis, welches das Ausf�hren von Anwendungen erlaubt, verwendet werden.

\subsection{Python 2.7}
\url{http://www.python.org/}

Als Interpreter f�r die verwendete Programmiersprache empfehlen wir die Referenzimplementierung \emph{CPython} in der \emph{Version 2.7}, wobei auch andere kompatible Python-Implementierungen verwendbar sein sollten, solange die weiteren Abh�ngigkeiten Kompatibilit�t zu diesen aufweisen.

Bez�glich der Python-Prozessorarchitekturen konnten die Intel 32-bit-Architektur (x86) sowie die AMD 64-bit-Architektur (x64) erfolgreich erprobt werden.

\subsection{PyQt 4.9}
\url{http://www.riverbankcomputing.co.uk/software/pyqt/download}

Das Benutzer-Interface verwendet Nokias Qt-Bibliothek. Die hierf�r notwendigen Python-Bindungen werden durch Riverbank Computing Limited bereitgestellt. F�r Windows existieren vorkompilierte Pakete, w�hrend Linux- und Mac-Nutzer die Bibliothek selbst kompilieren m�ssen, sofern nicht bereits vorgefertigte Pakete f�r das Betriebssystem existieren (bspw. das Paket \texttt{py27-pyqt4} f�r Mac OS X unter Verwendung von MacPorts).

Anzumerken ist hierbei, dass zur Verwendung (und eventuellen manuellen �bersetzung) der PyQt-Bibliothek Nokias Qt-Bibliothek neben weiteren Abh�ngigkeiten (hierzu sei an die Seiten des PyQt-Distributors verwiesen) auf dem System vorhanden sein m�ssen.

\subsection{Clingo (part of Potassco)}
\url{http://potassco.sourceforge.net/}

Das von der Universit�t Potsdam gepflegte Potassco-Projekt stellt Softwarewerkzeuge f�r ASP-Programmierung (Answer Set Programmung) bereit. \emph{ShoppingTour} verwendet \emph{Clingo}, welches die Verbindung von clasp, einem L�sungsproblem f�r logische Probleme, und Gringo, einem Konvertierer zur Erzeugung von variablen-freien logischen Problembeschreibungen, darstellt. Vorkompilierte Bin�rdateien f�r Windows, Mac OS/Darwin und Linux sind von der Anbieterseite ebenso zu erhalten wie der Software-Quellcode.
