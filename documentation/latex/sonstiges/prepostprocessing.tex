\section{Pre- und Postprocessing}
Um die Implementierung unserer Algorithmen zu verbessern und den Zustandsraum zu verkleinern, wird vor der Ausführung ein Preprocessing-Schritt und nach der Ausführung ein Postprocessing-Schritt durchgeführt.

\subsection{Preprocessing}
Als Eingabe erhält das Programm u.a. eine Adjazenzmatrix eines gewichteten, ungerichteten Graphen, in dem die Geschäfte auf Knoten und die Wege zwischen den Geschäften auf Kanten abgebildet werden. Das Gewicht einer Kante ist der Abstand zwischen zwei Geschäften bzw. die Kosten, die bei der Fahrt entstehen. Es ist auch möglich, dass zwischen Geschäften gar keine Verbindung existiert, was entweder bedeutet, dass das Geschäft nur über ein anderes Geschäft erreichbar ist, oder dass der Graph nicht zusammenhängend ist.

Falls der Graph nicht zusammenhängend ist, können die Knoten, die nicht vom Start- und Endknoten aus erreichbar sind, einfach verworfen werden. Diese sind dann nämlich nicht relevant für die Berechnung, weil sie in keiner validen Lösung auftauchen können. Sie würden den Suchraum nur unnötig vergrößern. Nicht erreichbare Knoten können mit einer Tiefensuche in linearer Zeit\footnote{$\mathcal{O}(|V|+|E|)$, wobei $|V|$ die Anzahl der Knoten und $|E|$ die Anzahl der Kanten ist.} gefunden und gelöscht werden.

Die eigentlichen Algorithmen entscheiden nur, ob und in welcher Reihenfolge bestimmte Knoten angesteuert werden sollen. Es wird also nicht direkt betrachtet, welchen Preis die Einkaufsartikel haben. Die Lösung, die ein Algorithmus generiert, ist nur eine Liste mit den anzusteuernden Knoten. Welche Artikel bei welchem Geschäft gekauft werden, muss nicht gespeichert werden. Dazu geht man einfach alle Artikel durch ein wählt für jeden Artikel das Geschäft unter den besuchten Geschäften aus, das den Artikel zum niedrigsten Preis anbietet.

Eine solche Rundreise kann aber sehr groß sein, da es passieren kann, dass ein Geschäft mehrmals besucht werden muss\footnote{Man denke z.B. an einen Fall, wo ein Geschäft A nur über ein einziges anderes Geschäft B erreichbar ist. Dann muss man dieses Geschäft B mindestens zweimal besuchen, einmal um zu A zu gelangen und einmal um A wieder zu verlassen. Analog lassen sich Fälle konstruieren, bei denen ein Geschäft beliebig oft besucht werden muss.}. Durch eine Optimierung im Preprocessing kann der Suchraum aber stark verkleinert werden.

Dazu wird der Graph in einen vollständigen Graphen umgewandelt. Jeder Knoten ist dann von jedem anderen Knoten aus direkt erreichbar. Als Kantengewicht einer hinzugefügten Kante wählt man den minimalen Abstand zwischen den beiden betroffenen Knoten. In einem solchen Graphen muss kein Knoten mehr als einmal besucht werden. Um einen Artikel in einem bestimmten Geschäft zu kaufen, reicht es aus, den Knoten ein einziges Mal zu besuchen. Von diesem Knoten aus ist aber jeder andere Knoten direkt erreichbar. Man steuert nun also nur Geschäfte an, in denen man auch etwas kaufen will. Fälle, in denen man einen Knoten nur ansteuert, um einen anderen Knoten zu erreichen, gibt es nicht mehr.

Der Suchraum kann durch diese Optimierung verkleinert werden, da nur noch zyklenfreie Rundreisen\footnote{Eine Rundreise ist natürlich auch ein Zyklus, aber dieser Zyklus zählt hier nicht.} betrachtet werden. Ein Algorithmus muss nur noch entscheiden, welche Geschäfte er überhaupt besuchen will und dann eine optimale TSP-Rundreise finden.

\subsection{Implementierung des Preprocessing}
Um den Graphen in einen vollständigen Graphen zu überführen, wird zunächst mit dem Floyd-Warshall Algorithmus der kürzeste Weg zwischen allen Paaren von Knoten berechnet\footnote{Laufzeit $\mathcal{O}(|V|^3)$ bei $|V|$ Geschäften.}. Es wird sowohl die Länge als auch der eigentliche Weg gespeichert. Die Adjazenzmatrix des Graphen kann nun mit den berechneten Längen vervollständigt werden.

\subsection{Postprocessing}
Die generierte Lösung eines Algorithmus kann Kanten benutzen, die im vollständigen Graphen zwar enthalten sind, nicht aber im ursprünglichen Graphen. Solche Kanten müssen im Postprocessing identifiziert werden und durch den kürzesten Pfad zwischen den beiden betroffenen Knoten ersetzt werden. Das ist mit wenig Aufwand möglich, weil während der Berechnung der minimalen Pfade im Floyd-Warshall Algorithmus auch die Knoten, aus denen ein Pfad besteht, gespeichert wurden.

Die grafische Oberfläche zeigt Zwischenlösungen der Algorithmen an. Kanten, die nur im vollständigen Graphen existieren, nicht aber im originalen Graphen, sind gestrichelt eingezeichnet. Die endgültige Lösung enthält keine solchen Kanten mehr. Stattdessen kann es sein, dass Knoten dort mehrmals besucht werden.
