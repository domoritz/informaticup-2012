\section{Ein NP-schweres Problem}
Bei dem vorliegenden Problem handelt es sich um NP-schweres Problem. Das ist deshalb interessant, weil es somit nahezu ausgeschlossen ist, einen Algorithmus zu finden, der die optimale L�sung des Problems effizient (in polynomieller Zeit) berechnet. Durch polynomielle Reduktion des \emph{Travelling Salesman Problems} (TSP) auf das vorliegende Problem kann die NP-Schwierigkeit gezeigt werden.

Beim TSP geht man von einem ungerichteten, gewichteten Graphen aus, wobei die Knoten des Graphen St�dte und die Kanten des Graphen Verkehrswege zwischen den St�dten repr�sentieren. Das Gewicht einer Kante gibt Aufschluss �ber die L�nge des Weges zwischen zwei St�dten. Nun soll eine Rundreise, also ein Pfad mit dem gleichen Start- und Endknoten, mit dem kleinstm�glichen Reiseweg gesucht werden, wobei alle St�dte einmal besucht werden m�ssen. 

Das TSP l�sst sich auf das Problem der Einkaufsplanung reduzieren, indem man den Graphen ohne Ver�nderung �bernimmt\footnote{Einkaufsgesch�fte sind St�dte.}. Wenn $S=\{s_1, s_2, \ldots\}$ die Menge der St�dte ist, erzeugen wir f�r jede Stadt $s_i$ einen Artikel $a_i$, den es nur in dieser Stadt zu kaufen gibt. F�r die letzte Stadt $s_n$ gibt es keinen Artikel, denn das soll unser Start- und Endpunkt sein, an dem die Reise beginnt. Der Preis der Artikel spielt keine Rolle, solange er endlich ist, also setzen wir ihn einfach auf $0$. 

Jeder Artikel soll nun einmal eingekauft werden. Durch die L�sung des Einkaufsplanungsproblems erhalten wir eine optimale L�sung des TSP, denn es werden die Gesamtkosten minimiert, die in diesem Fall nur aus den Reisekosten bestehen. Die Reisekosten aber ist die L�nge der Rundreise. Um jeden Artikel einzukaufen, muss des weiteren jede Stadt einmal besucht werden. Als Ergebnis erhalten wir u.a. die Reihenfolge, in der die St�dte besucht werden sollen, was der L�sung des TSP entspricht. Die Transformation des Problems ist trivialerweise in polynomieller Zeit m�glich, denn es muss lediglich f�r jede Stadt ein Artikel erstellt werden (linearer Zeitaufwand).

Um zu zeigen, dass es sich au�erdem um ein NP-vollst�ndiges Problem handelt, muss nachgewiesen werden, dass das Problem in der Menge NP enthalten ist. Es muss also gezeigt werden, dass eine L�sung in polynomieller Zeit verifiziert werden kann. An dieser Stelle soll nur das Entscheidungsproblem betrachtet werden: Gibt es eine L�sung des des Einkaufsplanungsproblems, deren Gesamtkosten unter dem Betrag $n$ liegen? Eine m�gliche L�sung kann durch Aufsummieren der Reisekosten zwischen den Einkaufsl�den\footnote{Die Reisekosten ergeben sich aus den Kantengewichten.} und dem Addieren der Artikelpreise bei den entsprechenden Einkaufsl�den in polynomieller Zeit\footnote{Wie man sich leicht �berlegen kann, macht es keinen Sinn, eine Kante mehr als zweimal zu benutzen.} verifiziert werden. Somit handelt es sich um ein NP-schweres Problem, das selbst in NP liegt, also folglich um ein NP-vollst�ndiges Problem.
