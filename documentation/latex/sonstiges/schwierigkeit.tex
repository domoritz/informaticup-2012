\section{Ein NP-schweres Problem}
Bei dem vorliegenden Problem handelt es sich um NP-schweres Problem. Das ist deshalb interessant, weil es somit nahezu ausgeschlossen ist, einen Algorithmus zu finden, der die optimale Lösung des Problems effizient (in polynomieller Zeit) berechnet. Durch polynomielle Reduktion des \emph{Travelling Salesman Problems} (TSP) auf das vorliegende Problem kann die NP-Schwierigkeit gezeigt werden.

Beim TSP geht man von einem ungerichteten, gewichteten Graphen aus, wobei die Knoten des Graphen Städte und die Kanten des Graphen Verkehrswege zwischen den Städten repräsentieren. Das Gewicht einer Kante gibt Aufschluss ¸ber die Länge des Weges zwischen zwei Städten. Nun soll eine Rundreise, also ein Pfad mit dem gleichen Start- und Endknoten, mit dem kleinstmˆglichen Reiseweg gesucht werden, wobei alle St‰dte einmal besucht werden m¸ssen.

Das TSP lässt sich auf das Problem der Einkaufsplanung reduzieren, indem man den Graphen ohne Veränderung übernimmt\footnote{Einkaufsgeschäfte sind Städte.}. Wenn $S=\{s_1, s_2, \ldots\}$ die Menge der Städte ist, erzeugen wir für jede Stadt $s_i$ einen Artikel $a_i$, den es nur in dieser Stadt zu kaufen gibt. F¸r die letzte Stadt $s_n$ gibt es keinen Artikel, denn das soll unser Start- und Endpunkt sein, an dem die Reise beginnt. Der Preis der Artikel spielt keine Rolle, solange er endlich ist, also setzen wir ihn einfach auf $0$.

Jeder Artikel soll nun einmal eingekauft werden. Durch die Lösung des Einkaufsplanungsproblems erhalten wir eine optimale Lösung des TSP, denn es werden die Gesamtkosten minimiert, die in diesem Fall nur aus den Reisekosten bestehen. Die Reisekosten aber ist die Länge der Rundreise. Um jeden Artikel einzukaufen, muss des weiteren jede Stadt einmal besucht werden. Als Ergebnis erhalten wir u.a. die Reihenfolge, in der die Städte besucht werden sollen, was der Lösung des TSP entspricht. Die Transformation des Problems ist trivialerweise in polynomieller Zeit möglich, denn es muss lediglich für jede Stadt ein Artikel erstellt werden (linearer Zeitaufwand).

Um zu zeigen, dass es sich außerdem um ein NP-vollständiges Problem handelt, muss nachgewiesen werden, dass das Problem in der Menge NP enthalten ist. Es muss also gezeigt werden, dass eine Lösung in polynomieller Zeit verifiziert werden kann. An dieser Stelle soll nur das Entscheidungsproblem betrachtet werden: Gibt es eine Lösung des des Einkaufsplanungsproblems, deren Gesamtkosten unter dem Betrag $n$ liegen? Eine mögliche Lösung kann durch Aufsummieren der Reisekosten zwischen den Einkaufsläden\footnote{Die Reisekosten ergeben sich aus den Kantengewichten.} und dem Addieren der Artikelpreise bei den entsprechenden Einkaufsläden in polynomieller Zeit\footnote{Wie man sich leicht überlegen kann, macht es keinen Sinn, eine Kante mehr als zweimal zu benutzen.} verifiziert werden. Somit handelt es sich um ein NP-schweres Problem, das selbst in NP liegt, also folglich um ein NP-vollst‰ndiges Problem.
