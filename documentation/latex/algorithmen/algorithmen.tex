\section{Optimierungsalgorithmen}

Nach dem Preprocessing wird auf die Daten ein Optimierungsalgorithmus angewendet, welcher iterativ bessere Lösungen findet und anschließend die beste gefundene präsentiert. Wir haben aus den Erfahrungen des letzten Informaticups und erneuten Analysen verschiedene Algorithmen evaluiert. Um die Komplexität nicht unnötig zu steigern wollten wir uns auf maximal zwei Algorithmen beschränken. Um trotzdem ausrechend Flexibilität zu behalten sollten die Parameter der Algorithmen aber so weit wie möglich anpassbar sein. Da die Probleminstanzen im Allgemeinen eine geringe Größe (ca. 10 Läden und ebensoviele Produkte) und wir den Suchraum durch unser Preprocessing schon weit einschränken konnten, haben wir uns dazu entschieden auch die Möglichkeit zur Berechnung von Optimalen Lösungen zu geben. 

Nachdem wir verschiedene Metaheurstiken wie \emph{Simulierte Abkühlung}, \emph{Greedy-Suche} und \emph{Tabusuche} untersucht haben, haben wir uns dazu entschieden einen \emph{genetischen Algorithmus} für die Lösung des Problems zu nutzen. Greedy-Suche hätte leicht in einem lokalen Minimum landen können, sodass dieser Algorithmus für uns ausschied. Simulierte Abkühlung hätte das Problem der lokalen Minima gelöst, aber die Performance von Genetischen Algorithmen ist im Allgemeinen höher. Die Tabusuche...

Da Meteheuristiken zwar oft sehr gute und manchmal auch optimale Lösungen finden können, aber diese nicht garantieren oder auch nur nachweisen können, haben wir einen weiteren Algorithmus implementiert. Das Problem der Optimalität liegt darin, dass Metaheuristiken mit Zufall arbeiten und den Suchraum nicht vollständig absuchen. Da das Problem aus der Aufgabenstellung NP-Schwer ist und selbst der Nachweis der Optimalität nicht trivial ist, kann es passieren, dass eine Metaheuristik nur eine unzureichend gute Lösung findet. Die Nutzung von Clingo ermöglicht es uns, in relativ kurzer Zeit sehr gute Lösungen zu finden und bei der letzten Lösung auch Optimalität zu garantieren! Damit ist unser Programm in diesem Punkt allen Implementierungen die nur auf Metaheuristiken aufbauen (wie auch unser genetischer Algorithmus), überlegen.

\subsection{ASP solver clingo}

\subsection{Genetischer Algorithmus}

Genetische Algorithmen lassen sich leicht auch das TSP-Problem anwenden und es wurde schon oft erläutert. Aus diesem Grund werden wir an dieser Stelle nicht weiter darauf eingehen, und verweisen für weitere Informationen auf \url{http://www.lalena.com/AI/Tsp/} verweisen. In diesem Abschnitt werden wir auf Besonderheiten unserer Implementierung eingehen. 

Eine dieser Besonderheiten ist die Darstellung der Individuen. Dabei wird nicht einfach eine Liste von besuchten Shops benutzt, da diese Repräsentation nach einem Crossover nicht notwendigerweise eine gültige Lösung generiert. So würde bei den naiven Liste \texttt{[0,1,3,2]} und \texttt{[0,3,1,2]} mit einem Crossover in der Mitte ein ungültiges Individuum \texttt{[[0,1,1,2]]} entstehen. Es ist ungültig, da die 1 zweimal besucht würde. 

Aus diesem Grund repräsentieren wir die Individuen als Liste, in der die einzelnen Werte für Indizes stehen. Beim Auflösen werden diese nacheinander an auf eine sortierte Liste angewendet und das jeweilige Element entfernt. Nehmen wir beispielsweise die interne Darstellung \texttt{[2,0,0]}. Nun soll diese in die die normale Repräsentation überführt werden. Dazu nehmen wir die sortierte Liste \texttt{[1,2,3]} und entfernen zuerst das dritte (Beginn bei 0, also Index 2). Nun haben wir eine Liste mit einem Element \texttt{[3]} und den Rest der sortierten Liste \texttt{[2,3]}. So verfahren wir weiter und erhalten die Liste \texttt{[3,1,2]}. Vorteil dieser Darstellung ist, dass sie beim Crossover immer eine gültige Lösung generiert und der Algorithmus wesentlich effizienter arbeiten kann.



