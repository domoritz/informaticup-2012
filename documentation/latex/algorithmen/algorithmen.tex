\section{Optimierungsalgorithmen}

Nach dem Preprocessing wird auf die Daten ein Optimierungsalgorithmus angewendet, welcher iterativ bessere Lösungen findet und anschließend die beste gefundene präsentiert. Wir haben aus den Erfahrungen des letzten Informaticups und erneuten Analysen verschiedene Algorithmen evaluiert. Um die Komplexität nicht unnötig zu steigern wollten wir uns auf maximal zwei Algorithmen beschränken. Um trotzdem ausrechend Flexibilität zu behalten sollen die Parameter der Algorithmen aber so weit wie möglich anpassbar sein. Da die Probleminstanzen im Allgemeinen eine geringe Größe (ca. 10 Läden und ebensoviele Produkte) aufweisen und wir den Suchraum durch unser Preprocessing schon weit einschränken konnten, haben wir uns dazu entschieden auch die Möglichkeit zur Berechnung von optimalen Lösungen zu implementieren. 

Nachdem wir verschiedene Metaheurstiken wie \emph{Simulierte Abkühlung}, \emph{Greedy-Suche} und \emph{Tabusuche} untersucht haben, haben wir uns dazu entschieden einen \emph{genetischen Algorithmus} für die Lösung des Problems zu nutzen. Greedy-Suche hätte leicht in einem lokalen Minimum landen können, sodass dieser Algorithmus für uns ausschied. Alle genannten Metaheuristiken, einschließlich dem genetischen Algorithmus, hätten das Problem der lokalen Minima gelöst, aber die Performance von genetischen Algorithmen ist im Allgemeinen höher. 

Da Meteheuristiken zwar oft sehr gute und manchmal auch optimale Lösungen finden können, aber diese nicht garantieren oder auch nur nachweisen können, haben wir einen weiteren Algorithmus implementiert. Das Problem der Optimalität liegt darin, dass Metaheuristiken mit Zufall arbeiten und den Suchraum nicht vollständig explorieren. Da das Problem aus der Aufgabenstellung NP-vollständig ist und selbst der Nachweis der Optimalität einer Lösung NP-vollständig ist, kann es passieren, dass eine Metaheuristik nur eine unzureichend gute Lösung findet\footnote{In unseren Tests funktionierte der genetische Algorithmus aber recht gut.}. Die Nutzung von Clingo ermöglicht es uns, in relativ kurzer Zeit sehr gute Lösungen zu finden und bei der letzten Lösung auch Optimalität zu garantieren! Damit ist unser Programm in diesem Punkt allen Implementierungen die nur auf Metaheuristiken aufbauen (wie auch unser genetischer Algorithmus), überlegen.

An dieser Stelle möchten wir noch anmerken, dass auch die Metaheuristik \emph{Simulierte Abkühlung} implementiert wurde. Jedoch nicht zur Berechnung einer Lösung des Problems sondern zur Verteilung der Knoten des Graphen auf der grafischen Ansicht. Der Algorithmus befindet sich in der Klasse \texttt{PositionCities} in der gleichnamigen Datei. Da die Funktionsweise des Algorithmus allgemein bekannt ist und der Algorithmus zudem nur zur grafischen Darstellung dient, möchten wir an dieser Stelle nur beschreiben, wie die Bewertung einer Lösung im Groben berechnet wird. Zu jeder Lösung wird für jedes Paar von Knoten der Abstand der Knoten auf der grafischen Ansicht mit dem Abstand der Knoten in der Adjazenzmatrix verglichen. Diese Abstände von grafischer Ansicht und Adjazenzmatrix verwenden subtrahiert, gewichtet und dann aufsummiert. Diesen Wert gilt es zu minimieren. Die Verteilung der Knoten mit diesem Algorithmus ist dann interessant, wenn es sich um echte, sinnvolle Abstände, d.h. um einen metrischen Graphen handelt. Bildet man beispielsweise ein existierendes Straßennnetz als Graphen ab und berechnet dann die grafische Darstellung mit diesem Algorithmus, erhält man ein \emph{sinnvolles} Abbild, wo sich z.B. Straßen nicht überschneiden, wenn an dieser Stelle keine Kreuzung ist. Und dazu muss die Position der Knoten nicht gespeichert werden, es wird lediglich die Graphstruktur benötigt.

Bei der Implementierung haben wir darauf geachtet, dass die Algorithmen als Generatoren/ Iteratoren implementiert sind. Das heißt, es kann ganz einfach von einer Lösung zu einer weiteren gewechselt werden. Des Weiteren kann die Fortsetzung leicht verzögert oder sogar abgebrochen werden.
\subsection{ASP solver clingo}

Clingo ist Programm, das ähnlich wie Prolog Logische Probleme löst. Zum vollständigen Verständnis dieses Abschnitts sind Grundlagen aus der Logikprogrammierung und KI erforderlich. Unabhängig vom Verständnis des Programms ist festzustellen, dass mit dieser Lösung Optimalität garantiert werden kann und gleichzeitig relativ kurz gerechnet wird. 

Die Stelle im Programm, an der die Implementierung zu finden ist, lautet \texttt{program/clingo.py}.

Das Eingabeformat von Clingo (bzw in diesem Fall Gringo) ähnelt dem von Prolog, wobei die Syntax erweitert wurde und auch Minimierungskriterien ermöglicht. Da die Programmbeschreibung von Clingo keine induktive wie Python, C++ oder Java ist, müssen auch andere Paradigmen angewendet werden. 

Clingo steht für \emph{clasp on Gringo} und kombiniert dadurch die beiden Systeme zu einem einfach zu verwendenden Programm. \emph{clasp} ist ein Problemlöser für erweiterte Logikprobleme. Es kombiniert abstrakte Modellierungsmöglichkeiten des \emph{answer set programming (ASP)} mit aktuellen lösungsalgorithmen aus dem Bereich des boolschen Constraint-Lösens. Dabei werden besonders die sehr effizienten konfliktgetriebenen Nachbarschaftssuchen verwendet. Dies ist eine Technik die sich in der Vergangenheit als sehr effizient herausgestellt hat. Clasp hängt dabei nicht von bestehenden Programmen ab sondern wurde von Grund auf neu entwickelt. Da clasp variablenfrei arbeitet, wird ein grounder benötigt, der eine bestehende Problemrepräsentation in eine gegroundete Version umwandelt. Ein solcher grounder ist Gringo. 

Für uns von Interesse sind allerdings weniger die technischen Details, als die tatsächliche Nutzbarkeit von Clingo für unser Problem. Da Clingo einen besonderen Teil unseres Programms darstellt, haben wir die leicht verständliche Dokumentation mit angefügt (\texttt{clingo\_guide.pdf}).

Unsere Problembeschreibung besteht aus vier Teilen. Der erste Teil ist die Definition des Graphs, wobei diese in Abhängigkeit vom Problem automatisch generiert wird. Ein Beispiel kann so aussehen.

\lstinputlisting{algorithmen/graph.lp}

Der zweite Teil ist die Definition der Kosten ,die ebenfalls automatisch generiert wird. Dabei werden die Pfadkosten E für die Reise von X nach Y als \texttt{expenses(X,Y,E)} und die Produktkosten C des Produkts P im Shop X als \texttt{cost(P,X,C)} dargestellt. \texttt{expenses(X,Y,E) :- expenses(Y,X,E).} definiert die Rückrichtung der Kanten. Kosten werden allgemein in Teilen von 100 Repräsentiert um Fließkommaoperationen zu verhindern. 

\lstinputlisting{algorithmen/cost.lp}

Der Dritte Teil ist die Definition der Lösung. \texttt{selected(X).} beschreibt, welche Nodes betrachtet werden, also in der Lösung auftauchen sollen. \texttt{cycle(X,Y)} beschreibt, dass es eine Verbindung von X nach Y gibt. \texttt{canReach(X,Y)} definiert Erreichbarkeit im Graphen. \texttt{minimum\_cost(P,CC)} definiert die geringsten Kosten CC für das Produkt P. Hier tritt die Optimierung zum Vorschein, dass wie nicht betrachten, wo ein Produkt gekauft wird, sondern nur, zu welchem Preis (Erklärung im Abschnitt Preprocessing). \texttt{first\_sorted\_cost}, \texttt{sorted\_costs} und \texttt{not\_minimum\_cost} sind Optimierungen für die Ermittlung der minimalen Kosten. Dabei wird vor dem eigentlichen Grounden der Lösung eine Sortierung der Preise erstellt, sodass die Suche anschließend um ein Vielfaches schneller wird. An sich hätte auch eine einfache Definition gerecht, die Beschreibt, dass es kein günstigeres gibt. Diese Version wäre allerdings um einiges langsamer. Am Ende wird mit der Definition von \texttt{bought} sichergestellt, dass auch wirklich jedes Produkt gekauft wird. 

\lstinputlisting{algorithmen/ham.lp}

Als letztes wird noch definiert, was minimiert werden soll.

\lstinputlisting{algorithmen/min.lp}

Anschließend muss Clingo nur noch mit den Beschreibungen als Eingabe ausgeführt werden und die Ausgabe entsprechend geparsed und Nachbearbeitet (Postprocessing) werden.

\subsection{Genetischer Algorithmus}

Beim vorliegenden Problem handelt es sich um eine leicht modifizierte Variante des TSP. Genetische Algorithmen lassen sich leicht auf das TSP-Problem anwenden. Genetische Algorithmen sind recht bekannt, deshalb wollen wir an dieser Stelle nur auf die Besonderheiten unserer Implementierung eingehen.

Eine dieser Besonderheiten ist die Darstellung der Individuen. Dabei wird nicht einfach eine Liste von besuchten Shops benutzt, da diese Repräsentation nach einem Crossover nicht notwendigerweise eine gültige Lösung generiert. So würde bei den naiven Liste \texttt{[0,1,3,2]} und \texttt{[0,3,1,2]} mit einem Crossover in der Mitte ein ungültiges Individuum \texttt{[[0,1,1,2]]} entstehen. Es ist ungültig, da die 1 zweimal besucht würde. 

Aus diesem Grund repräsentieren wir die Individuen als Liste, in der die einzelnen Werte für Indizes stehen. Beim Auflösen werden diese nacheinander an auf eine sortierte Liste angewendet und das jeweilige Element entfernt. Nehmen wir beispielsweise die interne Darstellung \texttt{[2,0,0]}. Nun soll diese in die die normale Repräsentation überführt werden. Dazu nehmen wir die sortierte Liste \texttt{[1,2,3]} und entfernen zuerst das dritte (Beginn bei 0, also Index 2). Nun haben wir eine Liste mit einem Element \texttt{[3]} und den Rest der sortierten Liste \texttt{[2,3]}. So verfahren wir weiter und erhalten die Liste \texttt{[3,1,2]}. Vorteil dieser Darstellung ist, dass sie beim Crossover immer eine gültige Lösung generiert und der Algorithmus wesentlich effizienter arbeiten kann.

Mutation tritt auf einem bestimmten Individuum mit einer bestimmten Wahrscheinlichkeit auf. Zuerst wird zufällig ein Gen, also ein Element der Liste ausgewählt. Dieses Element wird dann zufällig gesetzt. Die Anzahl der Gene, die pro Indiviuduum verändert werden, kann variiert werden. 

Eine Besonderheit in unserem genetischen Algorithmus ist das Auftreten von \emph{Katastrophen}. Bei einer Katastrophe wird eine große Anzahl an Individuen durch Mutation verändert, während die \emph{normale} Mutation nur sehr wenige Individuen betrifft. Das führt zwar dazu, dass unmittelbar nach der Mutation die Individuen insgesamt schlechter werden. Allerdings konnten wir feststellen, dass die endgültige Lösung dadurch etwas besser wird. 

Die Abbruchbedingung unseres genetischen Algorithmus ist das Durchlaufen einer großen Anzahl von Iterationen (Generationen), ohne dass ein besseres Individuum gefunden wurde. Der Vorteil gegenüber einer festen Anzahl an Iterationen liegt darin, dass der Algorithmus auch bei größeren Probleminstanzen gut funktioniert.

Der Code zum genetischen Algorithmus befindet sich in der Klasse \texttt{Genetic} in der Datei \texttt{program\_genetic.py}.


