\subsection{Genetischer Algorithmus}

Beim vorliegenden Problem handelt es sich um eine leicht modifizierte Variante des TSP. Genetische Algorithmen lassen sich leicht auf das TSP-Problem anwenden. Genetische Algorithmen sind recht bekannt, deshalb wollen wir an dieser Stelle nur auf die Besonderheiten unserer Implementierung eingehen.

Eine dieser Besonderheiten ist die Darstellung der Individuen. Dabei wird nicht einfach eine Liste von besuchten Shops benutzt, da diese Repräsentation nach einem Crossover nicht notwendigerweise eine gültige Lösung generiert. So würde bei den naiven Liste \texttt{[0,1,3,2]} und \texttt{[0,3,1,2]} mit einem Crossover in der Mitte ein ungültiges Individuum \texttt{[[0,1,1,2]]} entstehen. Es ist ungültig, da die 1 zweimal besucht würde. 

Aus diesem Grund repräsentieren wir die Individuen als Liste, in der die einzelnen Werte für Indizes stehen. Beim Auflösen werden diese nacheinander an auf eine sortierte Liste angewendet und das jeweilige Element entfernt. Nehmen wir beispielsweise die interne Darstellung \texttt{[2,0,0]}. Nun soll diese in die die normale Repräsentation überführt werden. Dazu nehmen wir die sortierte Liste \texttt{[1,2,3]} und entfernen zuerst das dritte (Beginn bei 0, also Index 2). Nun haben wir eine Liste mit einem Element \texttt{[3]} und den Rest der sortierten Liste \texttt{[2,3]}. So verfahren wir weiter und erhalten die Liste \texttt{[3,1,2]}. Vorteil dieser Darstellung ist, dass sie beim Crossover immer eine gültige Lösung generiert und der Algorithmus wesentlich effizienter arbeiten kann.

Mutation tritt auf einem bestimmten Individuum mit einer bestimmten Wahrscheinlichkeit auf. Zuerst wird zufällig ein Gen, also ein Element der Liste ausgewählt. Dieses Element wird dann zufällig gesetzt. Die Anzahl der Gene, die pro Indiviuduum verändert werden, kann variiert werden. 

Eine Besonderheit in unserem genetischen Algorithmus ist das Auftreten von \emph{Katastrophen}. Bei einer Katastrophe wird eine große Anzahl an Individuen durch Mutation verändert, während die \emph{normale} Mutation nur sehr wenige Individuen betrifft. Das führt zwar dazu, dass unmittelbar nach der Mutation die Individuen insgesamt schlechter werden. Allerdings konnten wir feststellen, dass die endgültige Lösung dadurch etwas besser wird. 

Die Abbruchbedingung unseres genetischen Algorithmus ist das Durchlaufen einer großen Anzahl von Iterationen (Generationen), ohne dass ein besseres Individuum gefunden wurde. Der Vorteil gegenüber einer festen Anzahl an Iterationen liegt darin, dass der Algorithmus auch bei größeren Probleminstanzen gut funktioniert.

